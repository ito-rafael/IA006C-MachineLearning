% \ll --> compile
% \lv --> view

% bold: 	\textbf{text}
% underline: 	\underline{text} 
% italic:	\textit{text}

% color:	\textcolor(color){text}



% align equations left
% change text color
% equation without number



\documentclass[a4paper]{article}
%-----------------------------
% preamble
%-----------------------------
\usepackage[sumlimits,]{amsmath}
\usepackage[utf8]{inputenc}
\usepackage[english]{babel}
%\usepackage[document]{ragged2e}
\usepackage{xcolor}

%-----------------------------
% body
%-----------------------------
\begin{document}

%=================================================
% Q1 - a)
\textbf{Exercício 1}
\\
a) Obtenha P(X) e P(Y).
\\
%------------------------
% P_X(0)
\textcolor{blue}{
\begin{equation}
	i) \; \underline{P_X(0):}
\end{equation}
}
\begin{equation}
	P_X(0) = \sum_{y} P_{XY}(0,y)
\end{equation}
\begin{equation}
	P_X(0) = P_{XY}(0,0) + P_{XY}(0,1)
\end{equation}
\begin{equation}
	P_X(0) = \frac{1}{6} + \frac{3}{8}
\end{equation}
\begin{equation}
	P_X(0) = \frac{13}{24}
\end{equation}
\\
%------------------------
% P_X(1)
\begin{equation}
	ii) \; P_X(1):
\end{equation}
\begin{equation}
	P_X(1) = \sum_{y} P_{XY}(1,y)
\end{equation}
\begin{equation}
	P_X(1) = P_{XY}(1,0) + P_{XY}(1,1)
\end{equation}
\begin{equation}
	P_X(1) = \frac{1}{8} + \frac{1}{3}
\end{equation}
\begin{equation}
	P_X(1) = \frac{11}{24}
\end{equation}
\\
%------------------------
% P_Y(0)
\begin{equation}
	iii) \; P_Y(0):
\end{equation}
\begin{equation}
	P_Y(0) = \sum_{y} P_{XY}(x,0)
\end{equation}
\begin{equation}
	P_Y(0) = P_{XY}(0,0) + P_{XY}(1,0)
\end{equation}
\begin{equation}
	P_Y(0) = \frac{1}{6} + \frac{1}{8}
\end{equation}
\begin{equation}
	P_Y(0) = \frac{7}{24}
\end{equation}
\\
%------------------------
% P_Y(1)
\begin{equation}
	iv) \; P_Y(1):
\end{equation}
\begin{equation}
	P_Y(1) = \sum_{y} P_{XY}(x,1)
\end{equation}
\begin{equation}
	P_Y(1) = P_{XY}(0,1) + P_{XY}(1,1)
\end{equation}
\begin{equation}
	P_Y(1) = \frac{3}{8} + \frac{1}{3}
\end{equation}
\begin{equation}
	P_Y(1) = \frac{17}{24}
\end{equation}
\\
%=================================================
% Q1 - b)
% P(X=0|Y=0)
b) Calcule P(X = 0|Y = 0).
\begin{equation}
	P_{X|Y}(x,y) = \frac{P_{XY}(x,y)}{P_Y(y)}
\end{equation}
\begin{equation}
	P_{X|Y}(0,0) = \frac{P_{XY}(0,0)}{P_Y(0)}
\end{equation}
\begin{equation}
	P_{X|Y}(0,0) = \frac{1/6}{7/24}
\end{equation}
\begin{equation}
	P_{X|Y}(0,0) = \frac{4}{7}
\end{equation}
\\
%=================================================
% Q1 - c)
c) Calcule E[X] e E[Y].
%------------------------
% E[X]
\begin{equation}
	i) \; E[X]:
\end{equation}
\begin{equation}
	E[X] = \sum_{k} x_k P_X(x_k)
\end{equation}
\begin{equation}
	E[X] = 0 \cdot P_X(0) + 1 \cdot P_X(1)
\end{equation}
\begin{equation}
	E[X] = \frac{11}{24}
\end{equation}
\\
%------------------------
% E[Y]
\begin{equation}
	ii) \; E[Y]:
\end{equation}
\begin{equation}
	E[Y] = \sum_{k} y_k P_Y(y_k)
\end{equation}
\begin{equation}
	E[Y] = 0 \cdot P_Y(0) + 1 \cdot P_Y(1)
\end{equation}
\begin{equation}
	E[Y] = \frac{17}{24}
\end{equation}
\\
%=================================================
% Q1 - d)
d) As variáveis são independentes? Por quê?
\\
\\
\\
TO BE WRITTEN
\\
\\
\\
%=================================================
% Q2 - a)
a) Calcule H(X), H(Y) e H(X|Y).
\begin{equation}
	P_X(0) = P_{XY}(0,0) + P_{XY}(0,1) = 1/4
\end{equation}
\begin{equation}
	P_X(1) = P_{XY}(1,0) + P_{XY}(1,1) = 3/4
\end{equation}
\begin{equation}
	P_Y(0) = P_{XY}(0,0) + P_{XY}(1,0) = 3/8
\end{equation}
\begin{equation}
	P_Y(1) = P_{XY}(0,1) + P_{XY}(1,1) = 5/8
\end{equation}
\\
%------------------------
% H(X)
\begin{equation}
	i) \; H(X):
\end{equation}
\begin{equation}
	H(X) = -\sum_{x} p(x) \log_2 [p(x)]
\end{equation}
\begin{equation}
	H(X) = -P_X(0) \log_2 [P_X(0)] -P_X(1) \log_2 [P_X(1)]
\end{equation}
\begin{equation}
	H(X) = - \frac{1}{4} \log_2 \left( \frac{1}{4} \right) - \frac{3}{4} \log_2 \left( \frac{3}{4} \right)
\end{equation}
\begin{equation}
	H(X) = 0.5 + 0.3113 = 0.8113
\end{equation}
\\
%------------------------
% H(Y)
\begin{equation}
	ii) \; H(Y):
\end{equation}
\begin{equation}
	H(Y) = -\sum_{y} p(y) \log_2 [p(y)]
\end{equation}
\begin{equation}
	H(Y) = -P_Y(0) \log_2 [P_Y(0)] -P_Y(1) \log_2 [P_Y(1)]
\end{equation}
\begin{equation}
	H(Y) = - \frac{3}{8} \log_2 \left( \frac{3}{8} \right) - \frac{5}{8} \log_2 \left( \frac{5}{8} \right)
\end{equation}
\begin{equation}
	H(Y) = 0.5306 + 0.4238 = 0.9544
\end{equation}
\\
%------------------------
% H(X,Y)
\begin{equation}
	iii) \; H(X,Y):
\end{equation}
\begin{equation}
	H(Y) = -\sum_x \sum_y p(x,y) \log_2 [p(x,y)]
\end{equation}
\begin{equation}
	H(Y) =  - P_{XY}(0,0) \log_2 [P_{XY}(0,0)] 
		- P_{XY}(0,1) \log_2 [P_{XY}(0,1)] 
		- P_{XY}(1,0) \log_2 [P_{XY}(1,0)] 
		- P_{XY}(1,1) \log_2 [P_{XY}(1,1)]
\end{equation}
\begin{equation}
	H(Y) = 	- 0 \cdot \log_2 (0) 
		- \frac{1}{4} \log_2 \left( \frac{1}{4} \right) 
		- \frac{3}{8} \log_2 \left( \frac{3}{8} \right)
		- \frac{3}{8} \log_2 \left( \frac{3}{8} \right)
\end{equation}
\begin{equation}
	H(Y) = 0 + 0.5 + 0.5306 + 0.5306 = 1.5613
\end{equation}
\\
%=================================================
% Q2 - b)
b) Calcule H(X|Y) e H(Y|X).
%------------------------
% H(X|Y)
\begin{equation}
	i) \; H(X|Y):
\end{equation}
\begin{equation}
	H(X,Y) = H(Y) + H(X|Y)
\end{equation}
\begin{equation}
	H(X|Y) = H(X|Y) - H(Y)
\end{equation}
\begin{equation}
	H(X|Y) = 1.5613 - 0.9544
\end{equation}
\begin{equation}
	H(X|Y) = 0.6068
\end{equation}
\\
%------------------------
% H(Y|X)
\begin{equation}
	ii) \; H(Y|X):
\end{equation}
\begin{equation}
	H(X,Y) = H(X) + H(Y|X)
\end{equation}
\begin{equation}
	H(Y|X) = H(X,Y) - H(X)
\end{equation}
\begin{equation}
	H(Y|X) = 1.5613 - 0.8113
\end{equation}
\begin{equation}
	H(Y|X) = 0.75
\end{equation}
\\
%=================================================
% Q2 - c)
% I(X,Y)
c) Calcule I(X,Y).
\begin{equation}
	I(X,Y) = H(X) - H(X|Y)
\end{equation}
\begin{equation}
	I(X,Y) = 0.8113 - 0.6068 = 0.2044
\end{equation}
\\
\begin{equation}
	I(X,Y) = H(Y) - H(Y|X)
\end{equation}
\begin{equation}
	I(X,Y) = 0.9544 - 0.75 = 0.2044
\end{equation}
\\
%=================================================
% Q3 - a)
%=================================================
% Q3 - b)









%\begin{equation}
%	e^{j\pi} + 1 = 0
%	\label{eqn:euler}
%\end{equation}

%\begin{equation}
%	\label{eqn:simple}
%	1 + 1 = 2
%\end{equation}

\end{document}
